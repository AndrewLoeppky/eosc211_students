 \documentclass[letterpaper,12pt]{article}
\usepackage{graphicx,amsmath,fancyhdr,times,mcode}
\oddsidemargin -0in %left margin on odd-numbered pages is 1in + 0in
\topmargin -.5in %top margin is 1in -0.5in

\textwidth 6.375in %width of text
\textheight 9in % size of page
\setlength{\parindent}{0.0in}
\setlength{\parskip}{10pt}
\renewcommand{\thefootnote}{\fnsymbol{footnote}}

\pagestyle{fancy}
\lhead{EOSC 211-2018}
\rhead{FUNCTION LAB -- \thepage} 
\chead{Week 8 - Algorithms-II (Functions)}
\lfoot{} 
\cfoot{} 
\rfoot{}

\newcounter{lnum}
\newenvironment{abbrevlist}%
  {\begin{list}{$\bullet$}{\setlength{\leftmargin}{2em}%
               \setlength{\itemindent}{0em}%
               \setlength{\itemsep}{0pt}%
               \setlength{\parsep}{0pt}%
               \setlength{\topsep}{2pt}%
               \usecounter{lnum} } }{\end{list}}


  
\begin{document}

\section*{Goals}
\begin{abbrevlist}
\item Turn your running median script into a function to compute a running median and implement various checks
\item  CODE:  Function definition and help lines
\item  CODE: subfunctions
\item  CODE: function calls from a main script
\item  CODE: Useful MATLAB functions:  \mcode{error},   \mcode{rem}, \mcode{size}
\end{abbrevlist}

Note:  Do NOT use the built-in MATLAB functions for computing running means and running medians in this lab.

%\section{Turning your Running Median Script into a Function}
\section*{The Lab}

%\subsection {The Running Median Script from Lab 6}
From the week 6 lab you should have a script to compute a running median and a running mean.  
One data set you used was contained in the file \mcode{aircraft_gps.mat}.  You'll use that file again here.
If your code didn't work you can use one of the posted solutions for that lab
from the web site.
If you use a posted solution convince yourself it works by plotting the
raw data as well as the running mean and running median (x-axis should be the variable \mcode{gps.mtime} which 
you can display in hrs, mins using  \mcode{datetick}).

%\begin{verbatim}
%% Sample script to calculate runnning mean and running median
%% Inputs:  gps.vel, winlen_mean, winlen_median
%% Outputs: z (running mean), zm (running median)
%% window length must be odd, loop sets window length to be smaller near ends of array 
%
%winlen_mean=7;    
%winlen_med=7;
%N=length(gps.vel);
%for i=1:N,
%    w1=min([i-1,N-i,(winlen_mean-1)/2]);  % also week 5 worksheet for min of 3 #s
%    z(i)=mean(gps.vel(i-w1:i+w1));	% running mean
%    w1m=min([i-1,N-i,(winlen_med-1)/2]);  
%    zm(i)=median(gps.vel(i-w1m:i+w1m));	%running median
%end
%\end{verbatim}

To make this code easier to use repeatedly, and to check for user input errors, we will a) turn
this code into a FUNCTION, and b) add some code that performs INPUT CHECKS to test for inappropriate inputs, embedded in the function itself.

\begin{enumerate}
\item{Function:  \mcode{calcmedian.m}}

Turn your \verb+runmean+ script into a function called \mcode{calcmedian} that takes ONLY two input arguments (\mcode{invec} and \mcode{winlen}, IN THAT ORDER), computes
ONLY the {\bf running median}, and returns ONLY \mcode{outvec}, a vector containing the
running median.  Provide comment lines to explain how you dealt with the ends of your array.  Make sure your function contains the H1 line,
indicating the function usage, and the input and output arguments.

\item{Script:  \mcode{testmedian.m}}

Write a short script called \mcode{testmedian} to (1) load the GPS data file, (2) call the running median function using 
input data \mcode{gps.vel} and a window length of 7, and (3) plot in the same figure the raw velocity data \mcode{gps.vel} and your running 
median (This script should be a lot like \mcode{runtest} from
the week 6 lab).  Check that the running median is working correctly
by examining your plot.

It is fun (and realistic) to use the aircraft GPS data to test your algorithm.  However, one problem with this is that rather subtle problems  
can get missed by a quick visual check because the data is somewhat noisy already.  You can (and should, if possible) also always test algorithms  
with test data that you make up yourself.  Here's an example of a test data set that we used to grade your week 6 lab script.  Try using it for debugging:

%\newpage

\begin{lstlisting}
y=[-1:.025:1];    % parabolic shape shows issues with window 'centering', 
                  % also shows scaling problems
invec=-y.^2;      % Predictable shape at ends makes problems in 
                  % end-handling easier to see
invec(1)=0;       % make end different (helps w/ end effects)
invec(40)=1;      % Single spike filtered out by median filter
invec([55:60])=1; % ...but longer step is not (for small window lengths)
                  % single spike and step also have 'nice' properties for 
                  % ...running mean
winlen=11;         
t=1:length(y);     
\end{lstlisting}

\item{Subfunction: \mcode{checkinputs}}

Write a subfunction called \mcode{checkinputs} that is called by \mcode{calcmedian} and does some INPUT CHECKS before starting 
with the calculations. Input checks are various tests on the inputs to see if they will cause problems in the later calculations. 
The code for \mcode{checkinputs}
should be physically written in the same file as \mcode{calcmedian} (see example of subfunction in your textbook).  The subfunction should print
a message indicating the nature of the problem and should stop the program if:
\begin{abbrevlist}
\item \mcode{invec} is not a column or row vector.  %Your error message should also print out the dimensions of \verb+invec+. 
\item \mcode{invec} is shorter than \mcode{winlen+1}.  % Your error message should also print out the values that were input for \verb+winlen+ and the length of \verb+invec+.
\item \mcode{winlen} is even.  % Your error message should also give the value of \verb+winlen+ used.
\item \mcode{winlen} is less than 3.  % Your error message should also give the value of \verb+winlen+ used.
\end{abbrevlist}
{\it Hint:} The built-in MATLAB functions \mcode{size} and \mcode{rem} might be useful in logical tests.

The built-in MATLAB function \mcode{error} is key for these kinds of checks, because it can be used to print an informative error
message AND stop the program. Type \mcode{help error} to understand why.
REMEMBER to test your code by trying out ``wrong'' scenarios -- e.g., check that you get the correct error message 
if you use an even \# for \mcode{winlen} in a call to \mcode{calcmedian}.

%\item{Function: \mcode{calcmean.m}}
%
%You can modify your \verb+calcmedian+ function to compute a running mean, using the built-in MATLAB function \mcode{mean}.   You don't need to 
%hand this in as part of this lab but it will be useful for Assignment \# 2.

\end{enumerate}
\section*{To Hand In}

Submit your function \mcode{calcmedian.m} via Connect by 4pm on Friday.  I will run it using a script that \
loads a data file and I will check that (1) the running median is implemented properly (I will check \
this by plotting the raw and running median data), (2) your function returns the 
appropriate error messages if I attempt to run it with the wrong input (e.g., if I attempt 
to use an even number for \mcode{winlen}).  


 As usual comment your code and add the following info AS A COMMENT at the bottom of the function header comment:
\begin{lstlisting}
%  partner.name=  'YYYYYY';
%  Time_spent=   XX   hours
\end{lstlisting}
 


\end{document}




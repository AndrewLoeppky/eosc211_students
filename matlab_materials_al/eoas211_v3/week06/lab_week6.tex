 \documentclass[letterpaper,11pt]{article}
\usepackage{graphicx,amsmath,fancyhdr,times,boxedminipage,wrapfig}
\usepackage{upquote,mcode,lineno}
\oddsidemargin -0in %left margin on odd-numbered pages is 1in + 0in
\topmargin -.5in %top margin is 1in -0.5in

\textwidth 6.375in %width of text
\textheight 9in % size of page
\setlength{\parindent}{0.0in}
\setlength{\parskip}{10pt}
\renewcommand{\thefootnote}{\fnsymbol{footnote}}

\pagestyle{fancy}
\lhead{EOSC 211-2018}
\rhead{LOOP LAB-\thepage} 
\chead{Week 6 - Algorithms-II (loops)}
\lfoot{} 
\cfoot{} 
\rfoot{}

\newcounter{lnum}
\newenvironment{abbrevlist}%
  {\begin{list}{$\bullet$}{\setlength{\leftmargin}{2em}%
               \setlength{\itemindent}{0em}%
               \setlength{\itemsep}{0pt}%
               \setlength{\parsep}{0pt}%
               \setlength{\topsep}{2pt}%
               \usecounter{lnum} } }{\end{list}}


  
\begin{document}
%%\linenumbers

\section*{Learning Goals}
%%\parbox{2in}{
In this lab, students will
\begin{abbrevlist}
\item develop algorithms involving loops.
\item carry out a top-down design process.
\item modularize code, separating the algorithm from input/output statements.
\item solve a problem useful in the earth sciences.
\end{abbrevlist}
%%}


%\begin{wrapfigure}[0]{r}{3.5in}
\begin{figure}[h]
\centering
\includegraphics[height=3in]{./spike1.eps}
\caption{ The aircraft dataset. The short spikes up and down, especially from 20:45 to 21:15,  are clearly erroneous.}
\label{f1}
\end{figure}
%%\end{wrapfigure}

\section*{Background}

In the very first lab you plotted a long sequence (a TIME SERIES) of 
temperatures at Sand Heads. 
Remember that the data showed a broad seasonal trend (colder in winter, warmer in summer), 
but that at very short time scales
(a day or two) there was a great deal of variability which gave the curve a fuzzy
look when you plotted a whole year. It is often very useful to be able to ``SMOOTH'' a time series to better display
broad trends by averaging away the short-time variability. 

A very simple way of
smoothing a time series is to use a so-called RUNNING MEAN. Imagine that $N$ data points are recorded hourly,
and number them 1,2,3,$\ldots$,$i-1,i,i+1$,$\ldots$,$N$ as they appear in a vector. 
If we are calculating a 5 point running
mean, then the 5th point in the SMOOTHED time series will be the average of points 3, 4, 5, 6 and 7 from
the original time series (from ``two-to-the-left" to ``two-to-the-right"). The 6th point in the smoothed time series will be the average of points
4, 5, 6, 7,and 8 from the original time series. The 7th point will be the average of points 5, 6, 7, 8, and 9.
And so on, moving over one point each time, with your window RUNNING through the time series, calculating the average of the points in 
the window at each point in the original series (Fig. \ref{f2}).

How de we describe this mathematically? Formally, we start with a time series vector $\bf x$, whose $i^{th}$ element is $x_i$ where
$i=1\ldots N$.
Then we form a new time series $\bf z$, again with $N$ elements, for which  the $i^{th}$ point $z_i$ is
an average of points in $x$ within the window. If the window has a length of 5, then
\begin{eqnarray}
z_i &=& \frac{ x_{i-2} + x_{i-1} + x_i + x_{i+1} + x_{i+2} }{5} \label{badsum} \\
    &=& \frac{1}{5} \sum_{k=-2}^{2} x_{i+k} 
\label{fsum}
\end{eqnarray}
and if it has a length $L$ (which has to be an odd number) then it would be
\begin{equation}
z_i = \frac{1}{L} \sum_{k=-W}^{W} x_{i+k}, ~~~\textrm{where}~~~W=\frac{(L-1)}{2}. 
\label{asum}
\end{equation}

\begin{figure}[t]
\centering
\includegraphics[height=1.2in]{./index.eps}
\caption{An example of the 5 point running mean. The $i$th element of z will be 9.2, an average of 1, 12, 21, 7, and 5.  The next
element will be 9.4, an average of 12, 21, 7, 5, and 2, and so on.}
\label{f2}
\end{figure}
 
In MATLAB  you can write  \mcode{x(i)} and \mcode{z(i)} as code elements corresponding to $x_i$ and $z_i$,  \mcode{x} for $\bf x$, etc., because
MATLAB was designed to translate math into code! Now,
we don't just program the long sum of 5 points in eqn. (\ref{badsum}), because we may want to change the length of
the window - we want $L$ as another input parameter. This is because we might want a 3 point running mean, or a 9 point
running mean. Instead, in this lab, you will use loops to program eqn. (\ref{asum}).


%Once the basic running mean algorithm has been designed, the code can be modified to do various other (useful)
%related things to time series. For example, the running mean can be replaced with 
%a  running WEIGHTED MEAN (where points towards the center of the window are weighted more than
%ones near the edges; this can provide a ``better looking'' smoothed curve), or with
%a running MEDIAN (to remove 'spikes' in noisy data), or you can use this basic 
%structure to estimate the derivatives of a noisy function (by least-squares fitting of a straight line to the points in
%the window), or count the hours per day that 
%temperatures are greater than 20$^{\circ}$C, etc. 

To write the required code, we will break the problem down into a series of steps, which you will follow
as you go through the lab:
\begin{abbrevlist}
\item First, develop a basic   algorithm, looping through all the points in a time
series of arbitrary length setting $x_i = <$something$>$.
\item Then, develop the code that computes the running mean inside that basic loop, so that
$x_i = <$running mean centered at point $i>$.
\item Next, add some \mcode{if}-statements to handle the awkward cases that take place at the start and end 
of the loop.
\item  Finally, copy/modify that code to handle a closely related (but different) task (the running median).
\end{abbrevlist}
Hint: it is good practice to write the code so that algorithm inputs and outputs are clearly 
defined using specific variable names. 

\section*{The Lab}

\begin{enumerate}

\item 
\begin{enumerate}

\item Start writing your code in a script (i.e. an m-file) \mcode{runmean.m}.  This script 
will itself
be called by another script \mcode{runtest.m}. \mcode{runtest} should look something like this:

\begin{boxedminipage}[h]{\linewidth}
\begin{lstlisting}
clear   % So no "junk" is left over from the last run!
% INPUTS
load lab1
x=temperature;   % input time series
winlen=25;       % the size of your window (an ODD number)

% CALCULATIONS
runmean;         % Now execute the code in runmean to do the work

% OUTPUTS
t=1:length(x);
figure(1);clf; hold on;
plot(t,x,'b');                   % original
plot(t,z,'.-r','linewi',1.5);    % Now add the output
\end{lstlisting}
\end{boxedminipage}
where the code in \mcode{runmean.m} will assume that \mcode{x} and \mcode{winlen} have already been defined,
and will produce a \mcode{z} which is then plotted with the code at the bottom of \mcode{runtest.m}.
Separating the running mean procedure from the definitions of inputs, and from output 
processing, is part of MODULARIZING code.

\item Within \mcode{runmean}, first write an OUTER loop in which the index (call it \mcode{i}) goes from 1 to \mcode{length(x)}. This outer loop is
meant to process every point in the time series.  

\item Now add an INNER LOOP (with index \mcode{k}) that implements the running mean itself. Note that MATLAB has
built-in \mcode{mean()} and \mcode{sum()} functions, but DO NOT use them here. You will run into problems
running your code. Why? Consider \emph{temporarily} changing the OUTER loop limits to something
that prevents these problems while debugging the code (but change it back for the next step).

\item Now add code to handle the END EFFECTS - those places at the beginning and the
end of the dataset where we don't have \mcode{(winlen-1)/2} points to the left (or right) of
the $i$th point - perhaps using \mcode{if} statements to perform the running mean differently.
Think of two different ways of handling these effects, and implement the
simplest (yes - think of two, but implement one). Describe in a comment how your TWO 
different solutions  would work, and why you
chose the one you chose.

\item In order to make really sure you have implemented the algorithm correctly, test
it on some data for which you can work out the answer. For example, set
\begin{lstlisting}
x=[1 5 3 7 9 8 4 6];
winlen=3;
\end{lstlisting}
Work out BY HAND what the answer should be, and then see if your program replicates this.

\item You can also test this on a real example, like the Sand Heads air temperature data from the lab in week 2 (called \verb+lab1.mat+). 
You may need to zoom
in on the plot to see the two series. If this is working properly the red line will be much
smoother than the blue line as daily variations are removed. Try different window lengths.

 
\end{enumerate}


\item Now write an algorithm to determine the running MEDIAN over a
window length \mcode{winlen_med} (use
the built-in function \mcode{median} to replace the running mean loop). Be careful and explicitly
label (with comments) what the inputs and outputs for the running median code should be.
The output variable for the running median code should be named \mcode{zm}.
Add this code to \mcode{runmean} so there are two outputs for the whole script: \mcode{z} and \mcode{zm}.
 
Why would you use a running median? In the case of the temperature data most (or all)
of the data is correct, or at least not wildly and obviously wrong. This is generally
untypical of real data, which often show very intermittent ``obviously
erroneous'' measurements for various reasons. For example, the figure
on page 1 shows a speed derived from GPS positions measured on a float-plane
flying a survey over the Strait of Georgia (this data can be found in  
the variable \mcode{gps.vel} contained in the structure \mcode{gps} which is stored in
the datafile \mcode{aircraft_gps.mat}). Note that the speed
tends to vary smoothly between 30 and 50~m/s, but there 
are strange spikes in the series (especially in the middle part of
the time series when the plane is turning a lot), related to changes in the paths of
radio-wave propagation when the aircraft changes its orientation and certain parts of the sky are obscured. 
These are clearly
``bad data'' because the speed of anything real won't change by 20~m/s for only one second!
 
 
Now, if we were interested in the speed, 
taking a running average will not really help matters (go ahead - use
your code above to see) because it will tend to pull the smooth curve away from the ``correct'' points.
But a running median, if the window is long enough,
can REJECT these OUTLIERS.  
 
Test your algorithm with \mcode{gps.vel} using \mcode{winlen_med=7}. When working 
properly the spikes will be removed from the time series.

 
\end{enumerate}





\section*{To Hand In (due 4pm Friday Oct 12)}

On CANVAS hand in your script file, called \mcode{runmean.m} exactly.  I will test it by running the
following code: 

\begin{boxedminipage}[h]{\linewidth}
\begin{lstlisting}
clear;
x=some_data_that_I_will_specify;            % NOT the aircraft data
winlen=some_length_not_necessarily_5;       % I'll use only odd numbersc
winlen_med=some other length;               % I'll use only odd numbers

runmean;

t=1:length(x);
figure(1); clf; hold on;
plot(t,x);       % Original time series
plot(t,z,'.-r')  % Running mean
plot(t,zm,'og'); % Running median
 \end{lstlisting}
\end{boxedminipage}
Include useful comments that describe your algorithm and also the following lines:
\begin{lstlisting}
partner.name='YYYYYY';
Time_spent= XX;
\end{lstlisting}
 
 


\end{document}




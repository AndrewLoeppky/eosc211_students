\documentclass[letterpaper]{article}
\usepackage{graphicx,amsmath,fancyhdr,times,url,hyperref}
\usepackage[pagewise]{lineno}
\usepackage{upquote,mcode}
\oddsidemargin -0in %left margin on odd-numbered pages is 1in + 0in
\topmargin -.5in %top margin is 1in -0.5in

%\ifpdf
%    \usepackage[pdftex]{graphicx} 
%    \usepackage{hyperref}
%    \pdfcompresslevel=0
%    \DeclareGraphicsExtensions{.pdf,.jpg,.mps,.png}
%\else
%    \usepackage{hyperref}
%    \usepackage[dvips]{graphicx}
%    \DeclareGraphicsRule{.eps.gz}{eps}{.eps.bb}{`gzip -d #1}
%    \DeclareGraphicsExtensions{.eps,.eps.gz}
%\fi


\textwidth 6.375in %width of text
\textheight 9.5in % size of page
 \setlength{\parindent}{0.0in}
 \setlength{\parskip}{7pt}
\renewcommand{\thefootnote}{\fnsymbol{footnote}}

\pagestyle{fancy}
\lhead{EOSC 211-2018}
\rhead{MATH LAB-\thepage} 
\chead{Week 4 - MATLAB Math}
\lfoot{} 
\cfoot{} 
\rfoot{}

\newcounter{lnum}
\newenvironment{abbrevlist}%
  {\begin{list}{$\bullet$}{\setlength{\leftmargin}{2em}%
               \setlength{\itemindent}{0em}%
               \setlength{\itemsep}{0pt}%
               \setlength{\parsep}{0pt}%
               \setlength{\topsep}{2pt}%
               \usecounter{lnum} } }{\end{list}}

\begin{document}

%%\noindent

 

In this lab we will evaluate mathematical formulae, and begin to use some
predefined MATLAB FUNCTIONS to assist in the computations. 

\section*{Goals}
 
\begin{abbrevlist}
\item Write a complete program with 4 parts:
\begin{abbrevlist}
\item Inputs
\item Internal definitions
\item Calculations
\item Outputs
\end{abbrevlist}
\item Write a complex program by continually modifying a simple program.
\item Learn some technical features of MATLAB
\begin{abbrevlist}
\item CODE: use some MATLAB built-in functions (Ch.~2, 11-28)
\item CODE: Perform simple mathematical operations using vector and scalar math (Ch.~3, 52-68)
\item evaluate a built-in function over a range of values (using \mcode{meshgrid})
\item plot the output of a built-in function with \mcode{legend} and \mcode{subplot}
\item adjust the axis limits using \mcode{axis} and \mcode{xlim}.
\end{abbrevlist}
 
 \end{abbrevlist}
  
 \begin{center} 
\includegraphics[width=3in]{./rfig}
\end{center}




\section{Predefined functions}

In class we practised with the built-in operators to multiply, divide, add,
and subtract. However, there are many other things that you might want to do
with one or more numbers. For example: rounding them, finding their sign, taking
their logarithm or exponential, or finding the remainder after integer division.
All of these (and more) can be done by calling a PREDEFINED FUNCTION. Here's an example:
\begin{lstlisting}
x=round( exp( sqrt( rem(z,3) ) ) );
\end{lstlisting}
This finds the remainder when a variable \mcode{z} is divided by 3 (i.e. returns a number less than
3), takes its square root, exponentiates the result, and finally rounds it to the nearest
integer. Which functions does MATLAB have? \mcode{help elfun} will give you a list.

{\bf Aside}: You don't need to remember the \mcode{elfun} list.   Matlab organizes its predefined functions in
``toolboxes'', each of which is in a specific directory. So if you know that
something called (say) \mcode{round} exists, you could try the \mcode{which} command to find it:
\begin{lstlisting}
>> which round
built-in (/localdata/matlab7/toolbox/matlab/elfun/@double/round)  % double method
\end{lstlisting}
and notice that it is in a directory \mcode{toolbox/matlab/elfun}, and try \mcode{help} as
above to see what else is there (the \mcode{@double}
tells you that \mcode{round} is a function of type \mcode{double} (or, in the jargon of object-oriented
programming a method of class \mcode{double}):
it rounds double precision floating point numbers.


\section{Clausius-Clapeyron equation and the atmospheric equation of state}

Over large spatial scales in the atmosphere (and ocean) we can attempt to predict the
winds (or currents) by looking at spatial changes in the DENSITY. In air the
density is a function of pressure, temperature, and the amount of water vapour 
present - i.e. the THERMODYNAMIC STATE\footnote{For oceanographers, density is a function of pressure, temperature,
and salinity, and for geologists the density of (say) molten rock will
depend on pressure, temperature, and its chemical composition. For this
lab you just have to know that there {\em are} equations to figure out!}. Moist
air is less dense than dry air, because $H_2O$ gas is lighter than either $N_2$ or $O_2$ gas (which
makes up most of the atmosphere).


We often hear water vapour measurements given as a RELATIVE HUMIDITY $RH$ which is the
the ratio of the vapor pressure relative to the SATURATION LEVEL for a particular temperature.
The saturation level is the maximum amount of water vapour that air can hold in equilibrium (say, inside 
a closed bottle containing air and water, which you let sit for a long time).
A $RH$ of 100\% means it is foggy (saturated)
and 0\% means you are coughing and generating sparks every time you walk across
a carpet! How do we get from $RH$ to density?


First, define the following quantities:
\begin{align}
\rho ~~& \text{Density,~kg~m$^-3$} \\
P ~~& \text{Pressure, Pa} \\
T ~~& \text{Temperature, $^{\circ}$K ~~~ (i.e., $^{\circ}$C+273.15}) \\
RH ~~& \text{Relative Humidity, \%} \\
\end{align}
as the things we are interested in, and
\begin{align}
r ~~& \text{mixing ratio, (g~g$^{-1}$), mass of water vapour to mass of dry air} \\
e ~~& \text{vapour pressure, Pa, which is usually less than:}\\
e_s ~~&  \text{saturation vapour pressure, Pa}\\
\end{align}
as temporary variables that we will need. Now, it would be nice to have (say)
a single, perhaps complicated, equation of state for density (as oceanographers do), but 
in fact atmospheric scientists define things using a chain of temporary 
variables\footnote{See, e.g., {\bf Meteorology Today for Scientists and
Engineers}, R. Stull}.
It all begins with the CLAUSIUS-CLAPEYRON equation which relates the temperature and the saturation vapour
pressure:
\begin{equation}
e_s = e_0 \cdot \exp \left[ \frac{L}{R_v} \left( \frac{1}{T_0} - \frac{1}{T}\right)\right]
\end{equation}
where $e_0 = 611$Pa, $L = 2.5\times10^6$J~kg$^{-1}$ is the latent heat of evaporation, $T_0 = 273$K is a constant (NOT
273.15K), and 
$R_v = 461$J~K$^{-1}$~kg$^{-1}$ is the gas constant for pure water vapour. Next we have
\begin{equation}
e = e_s \frac{RH}{100}
\end{equation}
and then
\begin{equation}
r = \frac{\varepsilon e}{P-e}
\end{equation}
where $\varepsilon = 0.622 {\rm kg_{vapour} kg_{dry air}^{-1}}$ ($=R_d/R_v$), and
finally
\begin{equation}
\rho = \frac{P}{ R_d T ( 1 + 0.61 r)}
\end{equation}
where $R_d = 287$~J~K$^{-1}$~kg$^{-1}$ is the ideal gas constant for dry air.\footnote{You may recognize some
similarity between this last equation and the IDEAL GAS LAW $PV=nRT$ which is no coincidence - in fact
the ``$0.61r$" is a correction for the non-ideal nature of moist air.} To check units, remember that 1 Pa = 1~kg~m$^{-1}$~s$^{-2}$, and 1 J=1~kg~m$^{2}$~s$^{-2}$.

%%\newpage

\section{Coding}
\label{sec:coding}

{\bf Four important points about programming style - most problems in this lab occur because
people don't read this!}:
\begin{itemize}
\item Be very careful with unit conversion.  Humans usually like to express temperature in Celsius,
but the thermodynamic equations require temperatures in Kelvin.  
That means that you need to distinguish between two temperatures
with names like \mcode{Tk} and \mcode{Tc}, where
\mcode{Tk = Tc  + 273.15}.  Make sure that you convert to Kelvin as soon as you can. 
Do it only once at the start and (if necessary for output) once at the end.

\item Choose good variable names. Don't use single letters like $T$, $e$ and $P$ for variable names
in a script.  It makes it very hard to do search and replace with your editor when you
want to change them, and the names aren't very meaningful to others who might read your code.  
Also remember that
\mcode{P} and \mcode{p} are two different variables to Matlab. 

\item As much as you can, separate your script into 4 parts IN THIS ORDER: 

\fbox{\parbox{6in}{
\begin{enumerate}
\item The INPUTS: These are variables whose values you might want to change if you run the program
multiple times.
\item the INTERNAL DEFINITIONS or CONSTANTS: These are variables that are defined as constants which you will
never change, or some simple conversions (e.g. Celcius to Kelvin)
\item the CALCULATIONS: This is where the complicated parts of the procedure go
\item the OUTPUTS: What you do once the math is finished (e.g.\ plotting).
\end{enumerate}
}}

Also

\fbox{\parbox{6in}{
It is useful to separate the parts by white space, and you should COMMENT your code! 
}}


Note that not all tasks in 2 and 3 are
easily separated, and sometimes plotting happens earlier than the end of the program, but it's a good idea to start thinking
about problems this way.

\item Finally, keep in mind that this lab is structured in such a way that you END UP solving a complicated problem, by FIRST solving a simple problem and then MODIFYING it. 

\end{itemize}


Now, given all of this:
\begin{enumerate}
\item Write a {\bf script} which defines variables for all the constants used
(i.e.~$L$, $R_v$, $T_0$, $e_0$) and then find
$e_s$ for a single $T$.  Be sure to invent good names for
all of these symbols and include their units in the comments, e.~g.:

\begin{verbatim}
Lvap = 2.5e6; % J/kg
\end{verbatim}

Run this code to find $e_s$ at a temperature of $T=25\ ^\circ C$. Compare this number against
a CHECK VALUE 3.264078243982589e+03 Pa. Wait, how do you see that many digits? \mcode{help format},
you want 'long'. Be careful about operator 
precedence and make sure you get the RIGHT answer!

\item Now MODIFY the code (that is, DO NOT append a whole new set of code repeating what is already done to
the bottom of your script) to  calculate
the air density at $T=25\ ^\circ C$, $RH=50\%$,
at surface pressure ($P=102000$ Pa). The answer should be 1.184710609882994 $\mathrm{kg\,m^{-3}}$ (or
1.184704073721439 if $\varepsilon\equiv R_d/R_v$). Note
that increasing the humidity (i.e. adding more water vapour) decreases 
the density since $H_2O$ is displacing heavier $N_2$ and $O_2$  molecules.

\item So far we have just dealt with one particular value. But it would be nicer
to plot the relationship.  Matlab can actually
generate a whole table of results for almost no extra time. The general strategy in writing code to do this will be to modify the definitions in the INPUT
section to make vectors or MATRICES
of \mcode{Tc}, \mcode{RH}, and \mcode{press} (assuming these are the names of input variables for temperature, relative humidity,
and pressure), instead of scalars, and then modify the CALCULATIONS to use the elementwise operators for multiplication and
division. 

To begin with plot the saturation vapour pressure for a range of
temperatures, using a code snippet something like this, with the \mcode{./} and \mcode{.*} operators:
\begin{lstlisting}
Tc=[-30:40]; %don't forget to convert to Kelvin for the calculations!
esat=  <your code here>
plot(Tc,esat); %...but plot against temperature in Celsius.
\end{lstlisting}
(note, \mcode{esat} is what I am calling $e_s$, you may have a different name...)

\item Next calculate and plot the mixing ratio $r$ as a function of temperature for $RH=100\%$
and $P=90000$ Pa (about 1 km above the surface).

\item Since we have a vector of \mcode{Tc} and scalars for \mcode{P} and \mcode{RH} we only
plot a single line. But it might be useful to plot mixing ratio (e.g., in a variable \mcode{mixrat}) for a variety of
\mcode{RH}. Instead of having to re-run the script using a different value 
each time for RH we can do it all in one step using  \mcode{meshgrid} in the INPUT section:
\begin{lstlisting}
[RH,Tc]=meshgrid([0:20:100],[-30:40]);
\end{lstlisting}
followed by the code you already have using elementwise operators.
Now this could compute a MATRIX \mcode{mixrat} (or whatever you call $r$), by redoing all the calculations, and then
\begin{lstlisting}
plot(Tc,mixrat)
xlabel('Temperature (^oC)');
ylabel('r (g/g)');
\end{lstlisting}
shows a plot very similar to the one on the front page. It is important to
understand what is going on here - investigate carefully 
what \mcode{meshgrid} does.

\item Now make a similar plot for density as a function of temperature, for a range
of relative humidities, again for \mcode{press}=90000 Pa.


\end{enumerate}



\section{To Hand In (due date is 4pm, Friday Sept. 28)}

Once you have done all of the items above, write a script file called \mcode{lab4.m} (and I mean
EXACTLY \mcode{lab4.m})\footnote{and not \mcode{lab\_4.m}, \mcode{mylab4.m}, \mcode{week4\_lab.m}, etc., etc....} which will
\begin{enumerate}
\item  Create a figure with two subplots for pressure \mcode{press}=90000 Pa: 
  \begin{enumerate}
     \item the upper showing mixing ratio as a function of temperature (in Celsius) for $RH=10, 30, 50,70$, and $90$\% (these
     percentages ONLY).
     \item the lower showing density as a function of temperature (in Celsius) for the same range of
        relative humidities.
  \end{enumerate}
\item Use the \mcode{legend} function to label the lines, and do \mcode{help legend} to see how to
change the legend location on the plot so it doesn't interfere with your lines.  Use \mcode{title}, 
\mcode{xlabel} and \mcode{ylabel} to label the plots,
and be sure to specify all of your units in your script.  
Using one or more of the functions \mcode{axis}, \mcode{xlim}, and \mcode{ylim}, {\bf limit the plot ranges} (on both plots) {\bf to} $5 \le T\  \le 25$ ($^\circ C$) and
$1.0 \le \rho \le 1.15$ ($\mathrm{kg\,m^{-3}}$) as applicable.


\item The script \mcode{lab4.m} should also contain the following  lines of working code:
\begin{lstlisting}
partner.name='YYYYYY';
Time_spent= XX ;
\end{lstlisting}
where YYYYYY is the person with whom you were paired in the labs, and
XX (a NUMBER) is your estimate of the HOURS spent both in the scheduled lab period and outside it to complete
this lab.\footnote{and I do mean ``working' code - don't just edit your file and submit it without testing; if you do there is 
a 20\% chance you have introduced an error that will make it crash when I mark it!}
\end{enumerate}

 


\end{document}




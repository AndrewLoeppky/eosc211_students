\documentclass[letterpaper]{article}
\usepackage{graphicx,amsmath,fancyhdr,times,lastpage}
\usepackage{color}
\oddsidemargin -0in %left margin on odd-numbered pages is 1in + 0in
\topmargin -.5in %top margin is 1in -0.5in

\textwidth 6.375in %width of text
\textheight 9.5in % size of page
\setlength{\parindent}{0.0in}
\setlength{\parskip}{10pt}
\renewcommand{\thefootnote}{\fnsymbol{footnote}}

\pagestyle{fancy}
\lhead{EOSC 211-2018}
\rhead{\thepage-\pageref{LastPage}} 
%\rhead{\thepage} 
\chead{Week 2: Interactive MATLAB - Introductory Lab}
\lfoot{} 
\cfoot{} 
\rfoot{}



\begin{document}


\section {Lab Overview}
The purpose of the first lab is to get you comfortable with the way MATLAB works and to show
you (by example) how to do a number of useful things.  

Nothing will be graded this week, but you must submit the item(s) noted in Section 10.  Future labs will assume you are familiar with
the material here. If you have computer experience some of this may seem
trivial but this will not necessarily be the case later on.  

\section{Pair Programming}
During labs you will be working in pairs using a technique called PAIR PROGRAMMING. It works like this:
\begin{enumerate}
\item You pair up with someone else 
\item One person (the ``driver'') sits in front of the computer and types. The other (the
``navigator'') watches what is
going on and makes comments.
\item Every 15 minutes, you SWITCH roles (we will announce the switchover time).
\item You are EACH responsible for handing in your own lab (you may or may not have identical labs). You MUST
add a comment to the code with the name of your partner.
\item If you have not completed the lab at the end of the lab period (which is usually the case),
you are free to a) schedule more pair-progamming time with your partner, or b) complete the lab
by yourself.
\end{enumerate}


%********************************
\section {Goals}
%********************************

\begin{itemize}
	\item familiarize yourself with the MATLAB interface and using it on a PC (or your own laptop)
	\begin {enumerate}
		\item logging on!
		\item learn how to make a folder (directory) and work IN THAT folder (directory) in MATLAB
	\end {enumerate}

	\item learn (by example) how to do a number of useful things within MATLAB, specifically how to
	\begin {enumerate}
		\item load and save data
		\item initialize and use variables
		\item make x-y plots of data: playing with colour, symbols, parts of your data set
		\item edit, save and run a basic MATLAB script
		\item comment your code
	\end {enumerate}
	
\end {itemize}

  
By the end of the lab you should be comfortable with the following MATLAB operations:

\begin {itemize}
\item \verb+load, save, =+: inputting / saving data and variables
\item \verb+plot+:  making simple plots
\item \verb+whos, clear, clf, close+:  listing \& clearing variables in the workspace, clearing and closing figures
\item \verb+help+:  MATLAB help!

\end {itemize}
 
\begin{figure}[h]
\centering
\includegraphics[height=2.8in]{./tfig.jpg}
\caption{}
\label{f1}
\end{figure}

\begin{figure}[h]
\centering
\includegraphics[height=2.8in]{./sandheadsmap.jpeg}
\caption{}
\label{f1}
\end{figure}

%********************************
\section{Starting MATLAB}
%********************************

\begin{itemize}
\item {\bf See also your text book: p. 3--6} 
\end{itemize}

{\it Organizing your Work:} 

If you are using a lab computer, then when you log on you will be provided with a virtual
drive for your own data. This is the \verb+Z:+ drive. In the \verb+Z:+  drive create a folder (also known as a directory) \verb+eosc211+ and then within that folder, another
sub-folder (sub-directory) \verb+lab_week02+.  
%You can later create a  \verb+Z:matlab+ directory is a useful place to (eventually) put programs that may be generally useful. 
The \verb+lab_week02+ folder will
be used to store files useful for this week's lab only (make other directories for 
future labs).
Change the CURRENT FOLDER above the command window
to \verb+Z:\eosc211\lab_week02+, as this will mean MATLAB will automatically read
from and write to that folder. You can do this via the pulldown menu, or
by typing the following in the command window:
\begin{verbatim}
>> cd Z:\eosc211\lab_week02
\end{verbatim}
 (Do not type the ``\verb+>>+" as this merely
indicates what the line will look like.  Also note that MATLAB itself is case-sensitive, {\it i.e.,} it matters 
whether you use upper or lower case.).  
{\it IMPORTANT:}  Make sure you are NOT saving your work to the \verb+C:+ drive.
This drive is specific only to the machine that you are working on right now, e.g., PC \#27,
which is NOT backed up.  

If you are running MATLAB on your own laptop, set up a \verb+eosc211+ sub-folder in some convenient folder and make a \verb+lab_week02+ sub-folder in \verb+eosc211+.  Change the current folder above the command window to this folder.

{\it TIP:}  In general there are at least two ways to do most things - via
the GUI interface (pull-down menus and so on), or via typing at the
COMMAND LINE\footnote{and there may be many different combinations of commands that
can give you the same result...}.  {\it e.g.,} you can see what files are in your current folder either by 
(a) looking at the file list in the ``Current Folder'' window (GUI approach), or (b) typing \verb+ls+ in the 
``Command Window'' (COMMAND LINE approach).   
You may use whichever method you want, but using the command
line interface is better in the long run as you will eventually
use these same commands as lines of code in your SCRIPTS.

{\it TIP:} Use the up/down-arrow keys to retrieve previous lines, and then edit them by
using the left/right arrows
if errors occur due to problems with syntax or semantics. 
 
 
%********************************
\section {Some Basics: creating, assigning and listing variables }

%********************************

\begin{itemize}
\item {\bf See also your text book: p. 6--11} 
\end{itemize}

Here we shall see how to create a variable name and assign a value to it.   In the command window type
\begin{verbatim}
>>  radius = 6371;
\end{verbatim}

Of course this is the radius of the Earth in km.  See {\bf class notes from Tuesday for an explanation of what has happened by doing this}.

Use the up arrow and the delete button to repeat the command without the semi-colon at the end.  What happens?

Now type 
\begin{verbatim}
>>  pi
\end{verbatim}

What happens?  It turns out that \verb+pi+ has a special meaning in MATLAB.  It is a 
{\it BUILT-IN FUNCTION} that returns the value
of $\pi$.   This is handy because you will never need to define
$\pi$ when you write MATLAB code.

Now type
\begin{verbatim}
>> 2*radius
\end{verbatim}

What happens?  

In both of the previous examples the result of a particular operation (calling the function \verb+pi+ or performing the
calculation \verb+2*radius+ ) is assigned to a default BUILT-IN variable \verb+ans+.  We will normally want to be more sophisticated than
this because we will want to later use the result of a particular operation -- i.e. we will want the result to be assigned to a new variable.
We do this as follows.  Type

\begin{verbatim}
>> diameter = 2*radius;
\end{verbatim}

and then look at the value assigned to diameter (how do you do this?).  Now check your workspace window and in the Command Window type

\begin{verbatim}
>> whos
\end{verbatim}

The command \verb+whos+ lists the variables in the Workspace. 

What does the command \verb+who+ do?

Variables can be of various types: real numbers, integers, characters.  We shall learn more about these data types in class {\bf next week}.



%********************************
\section {Creating, Editing and Saving A Matlab Script}

%********************************

\begin{itemize}
\item {\bf See also your text book: p. 75--80} 
\end{itemize}

%{\bf See section 2 of Chapter 2 of your text}

Here we shall see how to save your MATLAB commands in a script.  This is an essential part of programming, because
it ensures reproducibility.  If you were to quit MATLAB now (by typing ``quit'' in the Command Window or choosing 
{\bf MATLAB} $\rightarrow$ {\bf Quit Matlab}) neither your command history nor your variables \verb+radius+ and \verb+diameter+ will be saved.  

Do the following to create a MATLAB script:

1.  Choose {\bf New} $\rightarrow$ {\bf Script} from the Menu bar (notice the keyboard shortcut for future use) or the specific {\bf New Script} button if you have one (depends on your MATLAB version)

2.  An Editor window will appear.  In this type the following lines

\begin{verbatim}
% Script to calculate the surface area of the Earth in square kilometers
radius = 6371;
area = pi*radius*radius
\end{verbatim}

What does the \% sign do?

3.  Now save your script to a file by choosing  {\bf Save} $\rightarrow$ {\bf Save As} from the Menu bar.  Notice the choice of directories to which
you can save your file.  Save it in the \verb+lab_week02+ folder and name your file \verb+earthrad.m+.  Note that it's good programming practice not to use spaces or special characters in your filenames.

4.  In the previous section and in this script we saw
how to do a basic mathematical operation (multiplication \verb+*+).  Experiment with addition, subtraction, and division by creating
new variables called \verb+rad2+, \verb+rad3+, \verb+rad4+ that have values 5~km bigger than the Earth's radius, 20.3~km smaller and 25\% of Earth's radius respectively.
Add the lines of code that do this to your script, add a comment line that explains what they do, and save your script.
{\bf We will learn more about Computer Math in Week 4}.



%********************************
\section {Clearing the Workspace, Using Help, and Executing a MATLAB script}

%********************************

You can CLEAR the MATLAB workspace by typing
\begin{verbatim}
>> clear
\end{verbatim}

Do this and check the Workspace.  All the variables should have disappeared.
The easiest way of finding what \verb+clear+ does is to
type \verb+help clear+.

{\it NOTE:} There is also a \verb+clear workspace+ button in the GUI.  BE CAREFUL about using this by accident!

You can execute a MATLAB script by typing the name of the script in the Command Window.
({\it NOTE:} to be recognized as a MATLAB script you must save the file with the
subscript \verb+.m+).  Try this:

\begin{verbatim}
>> earthrad
\end{verbatim}

You should see your variables radius, area, rad2, rad3, rad4 and area in the Workspace.
Any lines in your script without a semicolon will result in the value of the variable being
printed to the Command Window.  Play around with this by deleting and adding semicolons
and re-executing your script.

You can also execute a MATLAB script in other ways - e.g. using the RUN button in the pull-down menu for the EDITOR window.  Try this.


%********************************
\section{Loading, plotting and saving data:  Air Temperature Example}
%********************************

%{\bf See Section 2.6 of the text}
\begin{itemize}
\item {\bf See also your text book: p. 89--101} 
\end{itemize}

In the download section of the course web pages is a link to a file
\verb+lab1.mat+. Copy this to your \verb+Z:\eosc211\lab_week02+ sub-folder.

A \verb+.mat+ file can contain one or more named variables. Load them in
using
\begin{verbatim}
>> load lab1
\end{verbatim}

You can see the size of these variables by clicking on the WORKSPACE tab, or, as before
by typing \verb+whos+

Note that you will see the variables previously defined, as well as some new
ones. If you CLEAR the workspace you can re-load \verb+lab1.mat+.

This dataset comprises hourly air temperature measurements at Sand Heads,
a weather station at the mouth of the Fraser River. There are 12213
measurements here (508 days worth). 


To see what is in these variables the obvious thing to 
do is to plot them.

Try typing

\begin{verbatim}
>> plot(temperature)
\end{verbatim}

How does your  figure compare with the one above? y-axis values?  x-axis values? labels?
It turns out that because you have not specified the x-axis, then the x-axis is just the
measurement number - i.e., an x-value of 100 means the 100'th measurement value.
As the temperature variable is an {\it array} 
of length 12213, the x-axis in the plot goes from 1 to 12213.  ({\bf We will cover arrays and data structures in class in Week 3.})

We can see a particular element of the array, {\it e.g., the 600'th point} (or element) by typing
\begin{verbatim}
>> temperature(600)
\end{verbatim}

and several points, {\it e.g., the 600'th point to the 609'th point} by typing
\begin{verbatim}
>> temperature(600:609)
\end{verbatim}

So, you can see that we could plot a subset of our data by typing
\begin{verbatim}
>> figure(2)
>> plot(temperature(2000:4000))
\end{verbatim}

Try it.  This plots the part of the data set that goes from point (element) 2000 to point (element) 4000
in a new figure window (Figure 2).
Note that unless you tell it to do otherwise (we shall see how later in the course), MATLAB will choose
the upper and lower limits of the y-axis automatically, so the scale on this plot might look quite different from
the scale in your plot in Figure 1.  Compare your Figures 1 and 2 and check that Figure 2 makes sense to you.

\vskip 10pt
{\bf Dates and times in the data set:}

Now try:
\begin{verbatim}
>> plot(time)
\end{verbatim}
Again there are 12213 values on the x-axis but what are the y-axis values?
This is a bit mysterious and is not obvious!!!!  MATLAB has an internal convention in which times are represented as decimal days since
the year 0 (so that 11:30 on May 29, 1953 is represented as 713468.47916666). 
We can convert the first of these 12213 numbers into something more human-readable using
the following:
\begin{verbatim}
>> datestr(time(1))
\end{verbatim}
If we want to use these times in a plot:
\begin{verbatim}
>> plot(time, temperature)
\end{verbatim}
we can convert the x-axis labels using
\begin{verbatim}
>> datetick('x',3)
\end{verbatim}

{\bf NOTE}: It's not always a good idea to just cut and paste the commands in the lab instructions into MATLAB because not all characters transfer correctly.  The \verb+'+  is an example - if you cut and paste you'll have to retype the \verb+'+ by hand in the command window.

{\bf TIP: It's not obvious exactly how the previous commands, \verb+datestr+ and \verb+datetick+, work.  However,  they are useful commands to store in your ``tool kit'' of useful MATLAB tricks.}
 
\vskip 10pt
{\bf Investigating Seasonal Temperatures}

We can investigate the distribution of these temperatures by plotting a histogram
\begin{verbatim}
>> hist(temperature)
\end{verbatim}
or perhaps with temperature bins that go from -2$^\circ$C to 30$^\circ$C (in steps of 1$^\circ$C).  Try:
\begin{verbatim}
>> hist(temperature,[-2:1:30]);
\end{verbatim}

Note also that if you don't supply the step interval MATLAB will assume it is a step of 1.   So
\begin{verbatim}
>> hist(temperature,[-2:30]);
\end{verbatim}

does exactly the same thing as the previous command.

Note that MATLAB will plot your new figure in whichever figure window was last active (Figure 2).  Remember
to use the arrows if you are repeating or just editing earlier commands.

It is interesting that the histogram shows two peaks. These are probably associated
with wintertime and summertime. We can extract summer and winter periods and overplot them
using
\begin{verbatim}
>>plot(time,temperature,time(5149:7308),temperature(5149:7308),...
                       time(9493:11700),temperature(9493:11700))
>>datetick('x',3)
\end{verbatim}

Note that the {\em ellipsis} (i.e. `\verb+...+') is punctuation
that means the next line is a continuation of the current one. The line is broken to fit
on the page here but you can put both lines together if you want.

This line is becoming complicated enough that typing (and retyping) quickly becomes
tedious.  Open a new script called \verb+plottemps.m+,
 type the line above into the script and execute your script.
You now see the entire time series, with winter data in reddish brown and the
second summer in yellow. What is the date of \verb+time(9493)+?

{\it TRY THIS:  PLOTTING USING DIFFERENT colours AND SYMBOLS:} Can we use other colours and symbols in the figure?  Yes!!  Use
\verb+help plot+, and then plot the time series in small red squares joined
by a dashed line, or in cyan triangles.

Now we can make a more complex plot which shows the distribution of temperatures in
summer and winter, as well as from the whole time series.  In your 
\verb+plottemps.m+ script, comment out the lines you have (i.e. insert a \verb+%+
at the beginning of each line), and type:
\begin{verbatim}
clf
subplot(311);
hist(temperature,[-2:30]);
subplot(312);  
hist(temperature(5149:7308),[-2:30]);
subplot(313);  
hist(temperature(9493:11700),[-2:30]);
\end{verbatim}
 
Execute your script.  What did \verb+clf+ do?
This plot shows that winter temperatures appear to be around $6^\circ$C, and
summer temperatures around $17^\circ$C, with some scatter above and below
both numbers.   The typing can be made easier (and less prone to error) by 
defining two new temperature arrays.  In your script, type:

\begin{verbatim}
wtemp=temperature(5149:7308);
stemp=temperature(9493:11700);
wtime=time(5149:7308);
stime=time(9493:11700);
plot(time,temperature,wtime,wtemp,stime,stemp)
datetick('x',3)
\end{verbatim}

Execute your script.  Notice that the new plot has appeared in the last subplot
of your Figure.  If you want the new plot to take up the whole figure again you must
type \verb+clf+ before plotting.  Try this.

You should see some new variables in your workspace window.
This would be particularly useful if we wanted to do more analyses on our subsets of winter
and summer temperatures: e.g., more statistics....

We can compute the mean (average) summer temperature using
\begin{verbatim}
>> mean(stemp)
\end{verbatim}
and the minimum using
\begin{verbatim}
>> min(stemp)
\end{verbatim}
Can you find the maximum temperature of the entire data set?  What about the maximum temperature for winter?

{\it TRY THIS: ADDING LABELS TO YOUR FIGURES:}  See if you can use the MATLAB help pages to figure out how to add a label to the
y-axis and give your plot a title.

Finally we have learned how to load a MATLAB data file (e.g., \verb+ load lab1.mat+).  We can save all or some of our variables
also. For example to save just our  summer temperature and time variables to a file you could type

\begin{verbatim}
>> save summer stime stemp
\end{verbatim}

This will save your summer times and temps to a file called \verb+summer.mat+.  Verify this by looking in your Current Folder window.

\section {Exiting MATLAB}

\begin {enumerate} 
\item Type ``quit'' in the Command Window,   OR  
\item Choose {\bf MATLAB} $\rightarrow$ {\bf Quit Matlab}.
\end{enumerate}

\section{To Hand In}

Make a new {\it script} called \verb+commute.m+. In it you will enter the distance you live from school in KM, your commute time in MINUTES, and 
a one-letter code for your mode of transport as follows:  W = walk, R = Ride bike, B = Bus, C = Car, O = other (horse/rollerblade/....) as follows:

\begin{verbatim}
% distance-km  commute-time-mins   transport-mode
       8.3              35				             R                             
\end{verbatim}

Use the EXACT format above.  Do NOT enter units
on the line that contains your data. Also, make sure to use spaces instead of tabs.

Save this file and upload on Canvas (http://canvas.ubc.ca/) {\bf by 4pm on FRIDAY}.   
     
\section {Remember to log out!}

\end{document}



